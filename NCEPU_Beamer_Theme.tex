\documentclass{beamer}
\usepackage{ctex}
\usepackage{hyperref}
\usepackage{calligra}
\usepackage[T1]{fontenc}
\usepackage{ncepu}
\usepackage{graphicx} % 插入图片
% \usepackage{subfig} % 子图
\usepackage[caption=false,font=tiny,labelfont=sf,textfont=sf]{subfig}
\usepackage{booktabs} % 三线表
\usepackage{multirow} % 三线表
\usepackage{caption}
% \usepackage[utf8]{inputenc} 
% \usepackage{fontspec}
% \setmainfont{Times New Roman}

\definecolor{ncepu_blue}{RGB}{0,90,160}

\newcommand{\itemEq}[1]{
\begingroup
\setlength{\abovedisplayskip}{0pt}
\setlength{\belowdisplayskip}{0pt}
\parbox[c]{\linewidth}{\begin{flalign}#1&&\end{flalign}}
\endgroup}

\author[林新辉]{答辩人:林新辉 \\ [5mm] 导师:李沂洹}
\title{基于数据驱动的动力电池健康状态估计和剩余寿命预测方法研究}
\subtitle{Research on Data-Driven Approaches for Estimating Health Status and Predicting Remaining Useful Life of Lithium-Ion Batteries in Electric Vehicles}
\institute{控制与计算机工程学院,华北电力大学}
\date{2023年6月13日}
\begin{document}

% 封面页
\kaishu
\begin{frame}
\titlepage
\end{frame}

% 目录页
\begin{frame}{\small 目录}
\tableofcontents[sectionstyle=show,subsectionstyle=show/shaded/hide,subsubsectionstyle=show/shaded/hide]
\end{frame}

\section{研究背景和研究对象}

\begin{frame}
	\begin{figure}[htbp]
		  \centering
		  \subfloat[SOH]   % 第一张子图的下标(注意:注释要写在[]中括号内)
		  {\label{fig:subfig1}\includegraphics[width=0.25\textwidth]{figures/nasa_B0005_failure_threshold.jpg}}
		  \subfloat[RUL]
		  {\label{fig:subfig2}\includegraphics[width=0.25\textwidth]{figures/RUL.png}}
		  \subfloat[SOC]
		  {\label{fig:subfig3}\includegraphics[width=0.25\textwidth]{figures/tri_b3c0_soc.jpg}}
		\captionsetup{font=tiny}
		\caption{电池SOH、RUL和SOC示意图}
		\end{figure}
	\begin{description}
		\item[SOH]
		电池健康状态,使用电池放电容量表征,描述电池性能退化状态,$SOH = \frac{Q_{max}}{Q_{nominal}}$
		\item[RUL]
		电池剩余寿命,描述电池从当前循环到寿命终止(EOL)循环的过程,$RUL = n_{EOL} - n_{current}$
		\item[SOC]
		电池荷电状态,和SOH有相同的形式,描述电池电荷量,$SOC = \frac{Q_{remain}}{Q_{max}}$
	\end{description}
\end{frame}

\section{建模和实验}

\subsection{基于电池容量历史退化数据的SOH估计}

\begin{frame}
	\begin{figure}[htbp]
		\centering
			\subfloat[CALCE\_CS2\_35电池上的SOH估计结果]
			{\label{fig:subfig1}\includegraphics[width=0.45\textwidth]{figures/soh_cap/slide_figure_calce.jpg}}
			\subfloat[NASA\_PCoE\_B0005电池上的SOH估计结果]
			{\label{fig:subfig2}\includegraphics[width=0.45\textwidth]{figures/soh_cap/slide_figure_nasa.jpg}}
	\captionsetup{font=tiny}
	\caption{五种模型的SOH估计结果示意图}
	\end{figure}
	\begin{itemize}
		\item 在CALCE数据集(包含4颗电池的循环数据)和NASA PCoE数据集(包含4颗电池的循环数据)上开展实验,采用留一法划分数据集
		\item 受篇幅限制,展示CALCE数据集中编号为CS2\_35电池和NASA PCoE数据集中编号为B0005电池上的测试结果
	\end{itemize}
\end{frame}

\begin{frame}
\begin{table}[]
	\centering
	\resizebox{\columnwidth}{!}{%
	\begin{tabular}{ccccccccccc}
	\toprule
	\multirow{2}{*}{评价指标} & \multicolumn{5}{c}{CALCE数据集}          & \multicolumn{5}{c}{NASA PCoE数据集}      \\
						& AR    & SVR   & MLP   & LSTM  & CNN   & AR    & SVR   & MLP   & LSTM  & CNN   \\
	\midrule
	平均MaxE              & \underline{0.116} & 0.141 & 0.147 & 0.152 & 0.142 & \underline{0.054} & 0.097 & 0.111 & 0.158 & 0.113 \\
	平均MAE               & 0.010 & 0.023 & 0.009 & 0.028 & \underline{0.008} & \underline{0.008} & 0.033 & 0.019 & 0.041 & 0.020 \\
	平均RMSE              & 0.016 & 0.028 & 0.014 & 0.035 & \underline{0.013} & \underline{0.013} & 0.037 & 0.026 & 0.057 & 0.027 \\
	\bottomrule
	\end{tabular}%
	}
	\captionsetup{font=tiny}
	\caption{五种模型预测性能评估结果}
	\end{table}
	\begin{itemize}
		\item 五个模型均取得较高预测精度,使用数据驱动方法实现锂离子电池SOH估计具有可行性
		\item 对于短期时间序列预测问题,非隐状态模型(AR、SVR、MLP和CNN)的预测精度优于隐状态模型(LSTM)
	\end{itemize}
\end{frame}

\subsection{基于电池充放电直接测量量的SOH估计}

\begin{frame}
	\begin{figure}[htbp]
		\centering
			\subfloat[VIT输入,无变换]
			{\label{fig:subfig1}\includegraphics[width=0.25\textwidth]{figures/soh_vitq/tri_group1_cell4_cnn_vit.jpg}}
		\subfloat[VIq输入,无变换]	
		{\label{fig:subfig2}\includegraphics[width=0.25\textwidth]{figures/soh_vitq/tri_group1_cell4_cnn_viq.jpg}}
		\subfloat[VIT输入,有变换]
			{\label{fig:subfig3}\includegraphics[width=0.25\textwidth]{figures/soh_vitq/tri_group1_cell4_cnn_vit_trans.jpg}}
		\subfloat[VIq输入,有变换]	
		{\label{fig:subfig4}\includegraphics[width=0.25\textwidth]{figures/soh_vitq/tri_group1_cell4_cnn_viq_trans.jpg}}
	\captionsetup{font=tiny}
	\caption{CNN模型在四种不同训练配置下SOH估计结果示意图}
	\end{figure}
	\begin{itemize}
		\item 在TRI数据集(包含16块电池的循环数据,均分为四组)上进行实验,对电池分组采用留一法划分数据集
		\item 受篇幅限制,展示其中编号为b3c0的电池在四种不同实验配置下的测试结果
	\end{itemize}
\end{frame}

\begin{frame}
\begin{table}[]
	\centering
	\resizebox{\columnwidth}{!}{%
	\begin{tabular}{ccccc}
	\toprule
	评价指标   & VIT输入,无变换     & VIT输入,有变换    & VIq输入,无变换    & VIq输入,有变换    \\
	\midrule
	平均MaxE & 0.110514 & 0.131351 & 0.08509  & 0.068284 \\
	平均MAE  & 0.011018 & 0.015671 & 0.006605 & 0.006807 \\
	平均RMSE & 0.016167 & 0.021842 & 0.010835 & 0.009845 \\
	模型参数量  & 60421    & 12693    & 60421    & 12693   \\
	\bottomrule
	\end{tabular}%
	}
	\captionsetup{font=tiny}
	\caption{四组实验CNN模型SOH估计性能评估结果}
	\end{table}
	\begin{itemize}
		\item 对比使用V、I、T为输入的情形,使用V、I、q为输入时模型预测性能有显著提升
		\item 使用时间序列-图像变换能在保持预测精度的前提下显著降低模型参数量
	\end{itemize}
\end{frame}

\subsection{电池RUL预测}

\begin{frame}
	\begin{figure}[htbp]
		\centering
			\subfloat[电池044的RUL预测结果]
			{\label{fig:subfig1}\includegraphics[width=0.3\textwidth]{figures/rul/unibo_lstm_rul_cycle_4.jpg}}
			\subfloat[电池039的RUL预测结果]
			{\label{fig:subfig2}\includegraphics[width=0.3\textwidth]{figures/rul/unibo_lstm_rul_cycle_5.jpg}}
			\subfloat[电池041的RUL预测结果]
			{\label{fig:subfig3}\includegraphics[width=0.3\textwidth]{figures/rul/unibo_lstm_rul_cycle_6.jpg}}
	\captionsetup{font=tiny}		
	\caption{DeepLSTM模型的RUL预测结果示意图}
	\end{figure}
	\begin{itemize}
		\item 在Unibo Powertools数据集(包含27块电池的循环数据)上进行实验,其中20块电池数据用作训练集、7块电池数据被用作测试集
		\item 受篇幅限制,展示其中编号为044、039和041的电池上的测试结果
	\end{itemize}
\end{frame}

\begin{frame}
	% Please add the following required packages to your document preamble:
% \usepackage{graphicx}
\begin{table}[]
	\centering
	\resizebox{\columnwidth}{!}{%
	\begin{tabular}{ccccccccc}
	\toprule
	评价指标    & 电池003    & 电池011    & 电池013    & 电池006    & 电池044    & 电池039    & 电池041    & 均值       \\
	\midrule
	RMSE  & 3.855191 & 3.102227 & 10.77393 & 10.12258 & 13.87137 & 8.883196 & 4.498271 & 7.872395 \\
	NRMSE & 0.035861 & 0.02924  & 0.181346 & 0.028673 & 0.047128 & 0.019219 & 0.051681 & 0.056164 \\  
	\bottomrule
	\end{tabular}%
	}
	\captionsetup{font=tiny}
	\caption{DeepLSTM模型电池Ah-RUL预测性能}
	\end{table}
	\begin{itemize}
		\item 实验结果表明使用数据驱动方法实现锂离子电池RUL预测具有可行性
	\end{itemize}
\end{frame}

\section{总结与展望}

\begin{frame}
	\begin{itemize}
		\item 总结
			\begin{itemize}
				\item 基于电池容量历史退化数据实现SOH估计,比较五种模型的预测性能
				\item 基于电池充放电直接测量量实现SOH估计,使用电荷量取代电池表明温度作为模型输入提高预测性能,使用时间序列-图像变换减少模型参数量
				\item 基于电池充放电直接测量量实现RUL预测,提出依据容量定义的Ah-RUL取代依据循环圈数定义的cycle-RUL
			\end{itemize}
		\item 展望
			\begin{itemize}
				\item 估计/预测模型改进
					\begin{itemize}
						\item 融合机理模型和数据驱动模型,提高模型性能
						\item 引入贝叶斯模型,实现对预测结果的不确定性度量以更好支持工业决策
						\item 引入迁移学习,提高模型泛化能力
					\end{itemize}
				\item 模型在嵌入式平台的部署:模型量化和模型转换
			\end{itemize}
	\end{itemize}
\end{frame}

\section{写在最后}

\subsection{代码可用性}

\begin{frame}
	\centering
	本课题相关代码已在github上开源,请见:

	\centering
	\url {https://github.com/hilinxinhui/battery_phm.git}

	\begin{figure}[htbp]
		\centering
		\includegraphics[scale=0.35]{figures/github_contribution_log.png}
	\end{figure}

	\begin{figure}[htbp]
		\centering
		\includegraphics[scale=0.15]{figures/github_contribution_log_2.png}
	\end{figure}

\end{frame}

\subsection{感谢各位老师}

\begin{frame}
	\LARGE \centering 汇报完毕,请各位老师批评指正!
\end{frame}

\end{document}
